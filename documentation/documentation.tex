\documentclass[times, utf8, proizvoljni, numeric]{fer}
\usepackage{booktabs}
\usepackage{url}
\usepackage{multirow}
\usepackage{makecell}
\usepackage{tablefootnote}

\renewcommand\theadfont{\bfseries}

\begin{document}

\nocite{*}

% TODO: Navedite naslov rada.
\title{Određivanje LCP polja korištenjem modificiranog algoritma SA-IS}

% TODO: Navedite vaše ime i prezime.
\author{Juraj Juričić, Leon Luttenberger i Luka Suman}

\maketitle

\tableofcontents

\chapter{Uvod}
Sufiksno polje je vrlo važna struktura podataka kod indeksiranja teksta. Može se koristiti za (egzaktnu i parcijalnu) usporedbu teksta ili kompresiju podataka, kao i za izgradnju kompleksnijih indeksa poput sufiksnog stabla. U svim ovim primjenama, uz sufiksno polje potrebno je i polje najduljih zajedničkih prefiksa (engl. \textit{longest common prefixes -- LCP}) \citep{fischer2011inducing}.

LCP polje se može konstruirati u linearnom vremenu koristeći algoritam koji su predložili Kasai et al.\citep{kasai2001linear}, ali taj algoritam se nije mogao uspoređivati s performansama brzih algoritama za izgradnju sufiksnog polja.

Budući da se te dvije strukture često koriste skupa, u svojem radu \citep{fischer2011inducing}, Fischer izlaže kako se brzi algoritam induciranog sortiranja može preinačiti da istovremeno gradi i LCP polje.

U sklopu ovoga rada napravili smo našu implementaciju tog algoritma te usporedili dobivene rezultate s obzirom na brzinu izvođenja i utrošak memorije.

\chapter{Opis algoritma}
Ulaz u algoritam jest tekst $T$ koji sadrži $n$ znakova. Pritom se podrazumijeva (ne dodaje se eksplicitno) terminalni znak $\$$ na kraju teksta $T$ (pozicija $n-1$). Sufiks $S_{i}$ jest podniz od $T$ koji sadrži znakove od pozicije $i$ (uključeno) do kraja teksta ($T_{i...n}$). Potrebno je izračunati sufiksno polje $SA$ i $LCP$ polje. $SA$ je jednake duljine kao ulazni tekst ($n$) i sadrži permutaciju indeksa $i$ sufiksa $S_{i}$ iz teksta $T$ u leksikografsko sortiranom poretku. Usporedba se radi znak po znak i kraći sufiksi su manji od duljih. $LCP$ polje je isto duljine $n$ i sadrži duljine najduljih zajedničkih prefiksa od sufiksa koji su susjedni u $SA$.

Prvi korak algoritma jest određivanje tipova svih sufiksa. Sufiks $S_{i}$ je tipa $S$ (engl. \textit{smaller}) ako leksikografski prethodi sljedećem sufiksu $S_{i+1}$, a inače je tipa L (engl. \textit{larger}). Dodatno, sufiks tipa $S$ je ujedno i tipa $S^*$ (još i oznaka $LMS$ od engl. \textit{left most S}) ako je sufiks lijevo od njega $L$ tipa. To je zapravo prvi $S$ u svakom mogućem uzastopnom nizu $S$-ova. Algoritam kreće od kraja teksta. Terminalni znak je tipa $LMS$, a sufiks lijevo od njega je $L$ tipa jer terminalni znak prethodi svim ostalima. Zatim, sufiks $S_{i}$ je tipa $S$ ako je $T_{i}$ $<_{lex}$ $T_{i+1}$ ili ako je $T_{i}$ $=_{lex}$ $T_{i+1}$ i $S_{i+1}$ je tipa. Inače, $S_{i}$ je tipa $L$ te ako je $S_{i+1}$ tipa $S$ onda $S_{i+1}$ postaje tipa $LMS$.

Zatim se koristi sortiranje po pretincima (engl. \textit{bucket sort}). $SA$ se podijeli u pretince koji sadrže sufikse koji počinju s istim znakom. Broj ponavljanja određenog znaka u $T$ određuje veličinu njegovog pretinca. Prvi pretinac je uvijek veličine $1$ jer pripada terminalnom znaku (koji prethodi svima). Svaki pretinac se sastoji od lijeve polovice koja će sadržavati $L$-sufikse te desne polovice koja će sadržavati $S$-sufikse ($LMS$ su isto $S$). $L$-sufiksi se dodaju na početak pretinca i svaki sljedeći se dodaje desno od zadnje dodanog $L$-sufiksa u istom pretincu. $S$-sufiksi se dodaju na kraj pretinca i svaki sljedeći se dodaje lijevo od zadnje dodanog $S$-sufiksa (ili $LMS$) u istom pretincu. Mjesto unutar pretinca gdje se susreću zadnji (skroz desni) $L$-sufiks i prvi (skroz lijevi) $S$-sufiks zove se $L/S$ šav (engl. \textit{L/S seam}).

$SA$ se prvo popuni s $LMS$-sufiksima gdje njihov prvi znak određuje u kojem pretincu će završiti. Zatim se pretinci u $SA$ popune s $L$-sufiksima koristeći inducirano sortiranje (engl. \textit{induced sorting}) u desno. Počevši od pozicije $i=0$ u sufiksnom polju $SA$, ako je sufiks $S_{SA[i]-1}$ tipa $L$ onda se $SA[i]-1$ stavi u $c$-pretinac ($c=T_{SA[i]-1}$). Na sličan način se induciranim sortiranjem u lijevo sortiraju $S$-sufiksi. Počevši od pozicije $i=n-1$, ako je $S_{SA[i]-1}$ tipa $S$ onda se $SA[i]-1$ stavi u $c$-pretinac. Ovaj postupak može pomiješati $S$ i $LMS$ sufikse (to je i cilj).

Sljedeći korak jest napraviti sažetak (engl. \textit{summary}) dobivenog $SA$. Počevši od početka $SA$, odredi se $LMS$-naziv svakog $LMS$-sufiksa. Pošto je prvi sufiks u $SA$ uvijek terminalni znak, on dobije $LMS$-naziv jednak $0$ (nazivi su zapravo brojevi). Sljedeći $LMS$-sufiks (ostali tipovi ne dobiju imena) $S_{SA[i]}$ dobiva novo ime ako se njegov $LMS$-podniz (engl. \textit{$LMS$ substring}) razlikuje od $LMS$-podniza prethodnog imenovanog $LMS$-sufiksa. $LMS$-podniz $LMS$-sufiksa $S_{i}$ jest $T_{i...j-1}$ gdje je $j$ pozicija sljedećeg $LMS$-sufiksa u $T$. Dobiveni sažetak $T'$ služi kao ulazni tekst u rekurzivni poziv algoritma. Kako originalni tekst $T$ može sadržavati najviše $n/2$ $LMS$-sufiksa, vrijedi $|T'|<|T|$. Rekurzija prestaje kada se dobiveni sažetak sastoji samo od jedinstvenih imena (brojeva). Tada se $SA'$ odmah dobije sortiranje po pretincima, a $LCP'$ sažetka izračuna se direktno iz $SA'$ korištenjem Kasai algoritma.

Dobiveni $SA'$ određuje točan poredak $LMS$-sufiksa što se iskoristi kako bi se ovaj put točno stavili u pretince u $SA$. Pritom se njihove $LCP'$ vrijednosti moraju skalirati jer se vrijednost $LCP'[k]$ odnosi na podniz $v_{SA'[k]...SA'[k]+LCP'[k]-1}$ unutar $T'$, što zapravo odgovara vrijednosti $\sum_{i=0}^{LCP'[k]-1}|R_{SA[k]+i}|$ koja je određena zbrojem duljina zajedničkih $LMS$-podnizova. Krajnjoj $LCP$ vrijednosti treba dodati i broj zajedničkih znakova nakon zadnjeg zajedničkog $LMS$-podniza.

Na kraj se opet provede ubacivanje $L$-sufiksa te zatim $S$-sufiksa kako bi se dobio konačan $SA$. Pritom se koriste sljedeći dodatni koraci za izračun $LCP$ vrijednosti.

Pretpostavimo da je algoritam ubacio $L$-sufiks $S_{SA[i]-1}$ u $c$-pretinac na poziciju $k$. Ako je to prvi $L$ sufiks u pretincu onda je $LCP[k]=0$. Inače postoji $L$-sufiks $S_{SA[i']-1}$ na poziciji lijevo ($k-1$) koji je stavljen u prethodnoj iteraciji $i'$. $i'$ i $i$ su pozicije sufiksa (to su sufiksi desno od njih u $T$) koji su inducirali sortiranje trenutnih sufiksa na pozicijama $k-1$ i $k$ . Ako su $i'$ i $i$ u različitim pretincima onda $S_{SA[i']}$ i $S_{SA[i]}$ imaju različite početne znakove te je $LCP[k]=1$ jer sufiksi $S_{SA[k-1]}$ i $S_{SA[k]}$ dijele samo početni znak (u istom su pretincu). Inače, ako su $i'$ i $i$ u istom pretincu onda je $LCP[k]=min(LCP[i'+1,i])+1$, a minimum na intervalu promatra samo vrijednosti koje su izračunate ($LCP$ se ne početku napuni s $-1$ i one se ne gledaju kod minimuma). Kod $S$ je isti postupak, razlika je što se gleda $k+1$ i interval je $LCP[i+1,i']$.

Kod postavljanja zadnjeg $L$-sufiksa u pretinac ($L/S$ šav) potrebno je naivnom usporedbom (znak po znak) ažurirati $LCP$ vrijednost između zadnjeg $L$-sufiksa i prvog sljedećeg $LMS$-sufiksa u istom pretincu (ako on postoji). Prilikom postavljanja zadnjeg (prvi lijevi) $S$-sufiksa u pretinac potrebno je naivnom usporedbom izračunati $LCP$ vrijednosti u odnosu na zadnji $L$ sufiks u istom pretincu (ako on postoji) koji se zapravo nalazi odmah lijevo.

\chapter{Primjer izvođenja algoritma}

Kao ulazni primjer koristit će se niz znakova \textit{baabaabac}. Ulaz se sastoji od 9 znakova kojem se implicitno dodaje terminalni znak (engl. \textit{sentinel}) $\$$ pa je zato $n=10$. Sufiksno polje $SA$ i $LCP$ su jednodimenzionalna polja s $n$ elemenata s početnim vrijednostima $-1$ (bilo koja negativna vrijednost). U nastavku su indeksi, znakovi i tipovi sufiksa u tabličnom prikazu.

\begin{center}
	\begin{tabular}{l | c c c c c c c c c c |}
		$i$ & 0 & 1 & 2 & 3 & 4 & 5 & 6 & 7 & 8 & 9 \\
		$T[i]$ & b & a & a & b & a & a & b & a & c & \$ \\
		$tip[i]$ & L & S* & S & L & S* & S & L & S* & L & S* \\
	\end{tabular}
\end{center}

Prvi korak jest stavljanje $LMS$-sufiksa u njihove pretince. Počevši s lijeva, prvi $LMS$ sufiks je na poziciji $1$ i započinje sa znakom $a$ pa on ide na kraj $a$-pretinca. Isto tako je i za $4$ i $7$, a $9$ ide na početak ($\$ <_{lex} \forall c$).

\begin{center}
\begin{tabular}{l | c | c c c c c | c c c | c |}
	& $\$$ & & & a & & & & b & & c \\ \hline
	$SA$ & -1 & -1 & -1 & -1 & -1 & -1 & -1 & -1 & -1 & -1 \\ \hline
	$1$ na $5$ & -1 & -1 & -1 & -1 & -1 & \textbf{1} & -1 & -1 & -1 & -1 \\
	$4$ na $4$ & -1 & -1 & -1 & -1 & \textbf{4} & 1 & -1 & -1 & -1 & -1 \\
	$7$ na $3$ & -1 & -1 & -1 & \textbf{7} & 4 & 1 & -1 & -1 & -1 & -1 \\
	$9$ na $0$ & \textbf{9} & -1 & -1 & 7 & 4 & 1 & -1 & -1 & -1 & -1 \\
\end{tabular}
\end{center}

Sljedeće se ubace $L$-sufiksi, ali na početak pretinaca. Prvo se gleda pozicija $i=0$ u $SA$. $SA[i]=9>0$ pa se gleda $S_{SA[i]-1}$ što je sufiks $c\$$ na poziciji $8$ u $T$. To je $L$-sufiks pa se stavlja u $c$-pretinac. $i=1,2$ se preskaču jer je $SA[i]=-1$. Za $i=3$ gleda se $S_{6}$ i to je $L$-sufiks koji ide u $b$-pretinac. Za $i=4$ u $SA[4]$ je vrijednost $4$. Sufiks $S_{3}$ je tipa $L$ pa se dodaje u $b$-pretinac. Zadnji $L$-sufiks dodajemo za $i=5$.

\begin{center}
	\begin{tabular}{l | c | c c c c c | c c c | c |}
		& $\$$ & & & a & & & & b & & c \\ \hline
		$SA$ & 9 & -1 & -1 & 7 & 4 & 1 & -1 & -1 & -1 & -1 \\ \hline
		$8$ na $9$ & 9 & -1 & -1 & 7 & 4 & 1 & -1 & -1 & -1 & \textbf{8} \\
		$6$ na $6$ & 9 & -1 & -1 & 7 & 4 & 1 & \textbf{6} & -1 & -1 & 8 \\
		$3$ na $7$ & 9 & -1 & -1 & 7 & 4 & 1 & 6 & \textbf{3} & -1 & 8 \\
		$0$ na $8$ & 9 & -1 & -1 & 7 & 4 & 1 & 6 & 3 & \textbf{0} & 8 \\
	\end{tabular}
\end{center}

$S$-sufiksi se dodaju na sličan način, jedino što se počinje od kraja. Pritom je važno uočiti da se mogu pomiješati $S$-sufiksi i $LMS$-sufiksi jer novi sufiksi gaze stare vrijednosti ($7$ preko $1$, $2$ preko $4$ i $5$ preko $7$).

\begin{center}
	\begin{tabular}{l | c | c c c c c | c c c | c |}
		& $\$$ & & & a & & & & b & & c \\ \hline
		$SA$ & 9 & -1 & -1 & 7 & 4 & 1 & 6 & 3 & 0 & 8 \\ \hline
		$7$ na $5$ & 9 & -1 & -1 & 7 & 4 & \textbf{7} & 6 & 3 & 0 & 8 \\
		$2$ na $4$ & 9 & -1 & -1 & 7 & \textbf{2} & 7 & 6 & 3 & 0 & 8 \\
		$5$ na $3$ & 9 & -1 & -1 & \textbf{5} & 2 & 7 & 6 & 3 & 0 & 8 \\
		$1$ na $2$ & 9 & -1 & \textbf{1} & 5 & 2 & 7 & 6 & 3 & 0 & 8 \\
		$4$ na $1$ & 9 & \textbf{4} & 1 & 5 & 2 & 7 & 6 & 3 & 0 & 8 \\
	\end{tabular}
\end{center}

$LMS$-sufiksi na pozicija $1$ i $4$ imaju jednake $LMS$-podnizove $aab$ pa dobiju isto ime ($1$). Sažetak je zatim $T'=1120$ gdje $0$ označava terminalni znak u sažetku (on se implicitno podrazumijeva). Rekurzivni poziv rješava reducirani problem te vraća polja $SA'$ i $LCP'$.

\begin{center}
	\begin{tabular}{l | c c c c |}
		$i$ & 0 & 1 & 2 & 3 \\
		$T'[i]$ & 1 & 1 & 2 & 0 \\
		$tip'[i]$ & S & S & S & S* \\
	\end{tabular}
\end{center}

\begin{center}
	\begin{tabular}{l | c | c c | c |}
		$SA'$ & -1 & -1 & -1 & -1 \\ \hline
		$LMS$ & \textbf{3} & -1 & -1 & -1 \\
		$L$ & 3 & -1 & -1 & \textbf{2} \\
		$S$ & 3 & \textbf{0} & \textbf{1} & 2 \\ \hline
	\end{tabular}
\end{center}

Samo je jedan $LMS$-sufiks u $T'$ pa je korak rekurzije trivijalan. Dobivaju se $SA''=[0]$ i $LCP''=[0]$. $SA'$ ostaje isti, a $LCP'$ ima vrijednost $1$ na poziciji 2 jer sufiksi $T'_{SA'[1]}$ i $T'_{SA'[2]}$ dijele jedan znak ($LMS$ naziv $1$) na početku. Preslikavanjem pozicija pomoću $SA'$ dobijemo konačan poredak $LMS$-sufiksa u $SA$ ($SA_{ekv.}[k]=i_{T}[SA'[k]]$). Skaliranjem $LCP'$ dobijemo $LCP$ vrijednosti $LMS$-sufiksa. $LCP'[2]=1$ što odgovara duljini $LMS$-podniza $|aab|=3$, ali još dodatno se provjeri da je $T_{1..4}=T_{4..7}=aabb$ pa je skalirana vrijednost $LCP'_{skal.}=3+1=4$. Iako je $LCP'[3]=0$ što znači da nema zajedničkih $LMS$-podnizova, dodatna pretraga nalazi zajednički znak $T_{4}=T_{7}=a$ pa je $LCP'_{skal.}=1$

\begin{center}
	\begin{tabular}{l | c c c c |}
		$T'$ & 1 & 1 & 2 & 0 \\
		$i_{T}$ & 1 & 4 & 7 & \textbf{9} \\ \hline
		$SA'$ & \textbf{3} & 0 & 1 & 2 \\
		$LCP'$ & 0 & 0 & 1 & 0 \\ \hline
		$SA_{ekv.}$ & \textbf{9} & 1 & 4 & 7 \\
		$LCP'_{skal.}$ & 0 & 0 & 4 & 1 \\
	\end{tabular}
\end{center}

Korištenjem informacija u $SA_{ekv.}$ i $LCP'_{skal.}$ dobiju se prve konačne vrijednosti za $SA$ i $LCP$.

\begin{center}
	\begin{tabular}{l | c | c c c c c | c c c | c |}
		& $\$$ & & & a & & & & b & & c \\ \hline
		$SA$ & \textbf{9} & -1 & -1 & \textbf{1} & \textbf{4} & \textbf{7} & -1 & -1 & -1 & -1 \\
		$LCP$ & \textbf{0} & -1 & -1 & \textbf{0} & \textbf{4} & \textbf{1} & -1 & -1 & -1 & -1 \\ \hline
	\end{tabular}
\end{center}

Opet se ubacuju $L$ te se pritom koriste dodatni koraci za računanje $LCP$ vrijednosti.

\begin{center}
	\begin{tabular}{l | c | c c c c c | c c c | c |}
		& $\$$ & & & a & & & & b & & c \\ \hline
		$SA$ & 9 & -1 & -1 & 1 & 4 & 7 & -1 & -1 & -1 & \textbf{8} \\
		$LCP$ & 0 & -1 & -1 & 0 & 4 & 1 & -1 & -1 & -1 & \textbf{0} \\ \hline
		$SA$ & 9 & -1 & -1 & 1 & 4 & 7 & \textbf{0} & -1 & -1 & 8 \\
		$LCP$ & 0 & -1 & -1 & 0 & 4 & 1 & \textbf{0} & -1 & -1 & 0 \\ \hline
		$SA$ & 9 & -1 & -1 & 1 & 4 & 7 & 0 & \textbf{3} & -1 & 8 \\
		$LCP$ & 0 & -1 & -1 & 0 & 4 & 1 & 0 & \textbf{5} & -1 & 0 \\ \hline
		$SA$ & 9 & -1 & -1 & 1 & 4 & 7 & 0 & 3 & \textbf{6} & 8 \\
		$LCP$ & 0 & -1 & -1 & 0 & 4 & 1 & 0 & 5 & \textbf{2} & 0 \\ \hline
	\end{tabular}
\end{center}

U zadnjem koraku dodaju se $S$-sufiksi i opet se koriste dodatni koraci za $LCP$. U predzadnjem koraku ispod $4$ je $-1$ jer se $LCP$ računa za sljedeću poziciju $k+1$. U zadnjem koraku smo došli na $L/S$ šav, ali nema $L$-sufiksa u pretincu pa je $LCP[1]=0$.

\begin{center}
	\begin{tabular}{l | c | c c c c c | c c c | c |}
		& $\$$ & & & a & & & & b & & c \\ \hline
		$SA$ & 9 & -1 & -1 & 1 & 4 & \textbf{7} & 0 & 3 & 6 & 8 \\
		$LCP$ & 0 & -1 & -1 & 0 & 4 & 0 & 0 & 5 & 2 & 0 \\ \hline
		$SA$ & 9 & -1 & -1 & 1 & \textbf{5} & 7 & 0 & 3 & 6 & 8 \\
		$LCP$ & 0 & -1 & -1 & 0 & 4 & \textbf{1} & 0 & 5 & 2 & 0 \\ \hline
		$SA$ & 9 & -1 & -1 & \textbf{2} & 5 & 7 & 0 & 3 & 6 & 8 \\
		$LCP$ & 0 & -1 & -1 & 0 & \textbf{3} & 1 & 0 & 5 & 2 & 0 \\ \hline
		$SA$ & 9 & -1 & \textbf{4} & 2 & 5 & 7 & 0 & 3 & 6 & 8 \\
		$LCP$ & 0 & -1 & -1 & \textbf{1} & 3 & 1 & 0 & 5 & 2 & 0 \\ \hline
		$SA$ & 9 & \textbf{1} & 4 & 2 & 5 & 7 & 0 & 3 & 6 & 8 \\
		$LCP$ & 0 & \textbf{0} & \textbf{4} & 1 & 3 & 1 & 0 & 5 & 2 & 0 \\ \hline
	\end{tabular}
\end{center}

\chapter{Mjerenja i rezultati testiranja}

Izlaz algoritma je ispravan na svim provjerenim ulaznim skupovima -- izlaz je uspoređen s izlazom originalne implementacije algoritma \citep{origimpl}.

Rezultati mjerenja brzine i utroška memorije vidljivi su u tablici \ref{mjerenja}. Pritom su mjerene performanse naše implementacije, originalne implementacije \citep{origimpl} te novije implementacije \citep{newimpl}. Razlika označava relativnu razliku performansi naše implementacije u odnosu na originalnu, budući da se performanse originalne i novije implementacije zanemarivo malo razlikuju.

Sva su mjerenja rađena na računalu s \textbf{CPU Intel Core i7-7700HQ}, \textbf{8 GB RAM} i \textbf{Linux Ubuntu 18.04}.

Sva mjerenja vremena su vršena 100 puta uzastopno te je rezultat uprosječen. Maksimalno relativno odstupanje među mjerenjima trajanja izvođenja nikad nije bilo veće od 10\%.

Mjerenja utroška memorije vršena su pomoću programskog paketa \textit{Valgrind} i \textit{Massif heap profiler} te je u obzir uzimana maksimalna potrošnja memorije (engl. \textit{peak memory usage}).

Iz mjerenja je vidljivo da naša implementacija algoritma postaje lošija što je alfabet veći: za manje alfabete (do 5 znakova -- prva 4 skupa) je brzina izvođenja i utrošak memorije unutar 100\% od originalne implementacije. S druge strane, s većim alfabetom (A-Z) utrošak memorije prelazi 100\%.

\pagebreak

\begin{table}[]
\begin{tabular}{|l|l|l|l|l|l|l|}
\hline
\thead{Ulazni skup}             & \thead{Veličina \\ datoteke} & \thead{Naša \\ impl.} & \thead{Orig. \\ impl.} & \thead{Novija \\ impl.} & \thead{Razlika} &                               \\ \hline
\multirow{2}{*}{E. coli} & \multirow{2}{*}{4.797.285}
   & 1,16  & 0,60  & 0,61  & 92\%  & \thead{Vrijeme (s)} \\ \cline{3-7} 
 & & 85,77 & 43,18 & 43,18 & 98\%  & \thead{Memorija (MB)} \\ \hline
\multirow{2}{*}{A. hydrogenalis} & \multirow{2}{*}{2.482.366}
   & 0.38  & 0,24  & 0,23  & 63\%  & \thead{Vrijeme (s)} \\ \cline{3-7} 
 & & 43,45 & 22,35 & 22,35 & 94\%  & \thead{Memorija (MB)} \\ \hline
\multirow{2}{*}{N. crassa} & \multirow{2}{*}{575.384}
   & 0,073 & 0,057 & 0,056 & 28\%  & \thead{Vrijeme (s)} \\ \cline{3-7} 
 & & 10,28 & 5,18  & 5,18  & 94\%  & \thead{Memorija (MB)} \\ \hline
\multirow{2}{*}{Synthentic A-E\tablefootnote{Skup nasumično generiranih znakova alfabeta A-E}} & \multirow{2}{*}{1.000.000}
   & 0,20  & 0,14  & 0,13  & 43\%  & \thead{Vrijeme (s)} \\ \cline{3-7} 
 & & 18,97 & 9,00  & 9,00  & 110\% & \thead{Memorija (MB)} \\ \hline
\multirow{2}{*}{Synthentic A-Z\tablefootnote{Skup nasumično generiranih znakova alfabeta A-Z}} & \multirow{2}{*}{500.000}
   & 1,51  & 0,63  & 0,63  & 139\%  & \thead{Vrijeme (s)} \\ \cline{3-7} 
 & & 105,2 & 45,00 & 45,00 & 134\%  & \thead{Memorija (MB)} \\ \hline
\multirow{2}{*}{Shakespeare \citep{shakespeare}\tablefootnote{Zbir svih Shakespeareovih djela \citep{shakespeare}}} & \multirow{2}{*}{5.458.199}
   & 1,29  & 0,55  & 0,55  & 134\%  & \thead{Vrijeme (s)} \\ \cline{3-7} 
 & & 100,5 & 49,13 & 49,13 & 105\%  & \thead{Memorija (MB)} \\ \hline
\end{tabular}
\caption{Rezultati mjerenja}
\label{mjerenja}
\end{table}

\chapter{Zaključak}

U sklopu ovoga rada implementirali smo modificirani algoritam induciranog sortiranja za određivanje LPC polje.
Modificirani algoritam se pokazao ispravnim na svim provjerenim ulaznim skupovima.

Kao što je vidljivo iz rezultata mjerenja, iako naša implementacija dobro radi na stvarnim ulaznim skupovima, na sintetskim skupovima prekoračuje vremensku i memorijsku razliku od $100\%$ u usporedbi s originalnom implementacijom.
Također, pokazano je kako se brzina izvođenja i utrošak memorije povećava s većim alfabetima.

Implementacija se može poboljšati efikasnijom manipulacijom memorije, odnosno ponovnim iskorištavanjem već alocirane memorije.
Veliki utrošak memorije i broj alokacija stvara česte promašaje u priručnoj memoriji (\textit{cache miss}), što također povećava vrijeme izvođenja programa.

\bibliography{documentation}
\bibliographystyle{fer}

\chapter{Sažetak}

U sklopu ovoga rada implementiran je i objašnjen modificirani algoritam SA-IS za  određivanja LCP polja.
Rad algoritma prikazan je na jednostavnom primjeru, te su performanse napisane implementacije uspoređene s originalnom implementacije.

\end{document}
