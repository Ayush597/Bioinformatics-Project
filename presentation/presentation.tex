%%%%%%%%%%%%%%%%%%%%%%%%%%%%%%%%%%%%%%%%%
% Beamer Presentation
% Juraj Juričić
% Faculty of Electrical Engineering and Computing
% University of Zagreb
%
% Project Presentation
%
%
%%%%%%%%%%%%%%%%%%%%%%%%%%%%%%%%%%%%%%%%%

%----------------------------------------------------------------------------------------
%	PACKAGES AND THEMES
%----------------------------------------------------------------------------------------

\documentclass{beamer}
\usepackage{pdfpages}

\usepackage{multirow}
\usepackage{makecell}
\usepackage{tablefootnote}

\mode<presentation> {

% The Beamer class comes with a number of default slide themes
% which change the colors and layouts of slides. Below this is a list
% of all the themes, uncomment each in turn to see what they look like.

%\usetheme{default}
%\usetheme{AnnArbor}
%\usetheme{Antibes}
%\usetheme{Bergen}
%\usetheme{Berkeley}
%\usetheme{Berlin}
%\usetheme{Boadilla}
%\usetheme{CambridgeUS}
%\usetheme{Copenhagen}
%\usetheme{Darmstadt}
%\usetheme{Dresden}
%\usetheme{Frankfurt}
%\usetheme{Goettingen}
%\usetheme{Hannover}
%\usetheme{Ilmenau}
%\usetheme{JuanLesPins}
%\usetheme{Luebeck}
\usetheme{Madrid}
%\usetheme{Malmoe}
%\usetheme{Marburg}
%\usetheme{Montpellier}
%\usetheme{PaloAlto}
%\usetheme{Pittsburgh}
%\usetheme{Rochester}
%\usetheme{Singapore}
%\usetheme{Szeged}
%\usetheme{Warsaw}

% As well as themes, the Beamer class has a number of color themes
% for any slide theme. Uncomment each of these in turn to see how it
% changes the colors of your current slide theme.

%\usecolortheme{albatross}
%\usecolortheme{beaver}
%\usecolortheme{beetle}
%\usecolortheme{crane}
%\usecolortheme{dolphin}
%\usecolortheme{dove}
%\usecolortheme{fly}
%\usecolortheme{lily}
%\usecolortheme{orchid}
%\usecolortheme{rose}
%\usecolortheme{seagull}
%\usecolortheme{seahorse}
%\usecolortheme{whale}
%\usecolortheme{wolverine}

%\setbeamertemplate{footline} % To remove the footer line in all slides uncomment this line
%\setbeamertemplate{footline}[page number] % To replace the footer line in all slides with a simple slide count uncomment this line

%\setbeamertemplate{navigation symbols}{} % To remove the navigation symbols from the bottom of all slides uncomment this line
}

\usepackage{graphicx} % Allows including images
\usepackage{booktabs} % Allows the use of \toprule, \midrule and \bottomrule in tables
\usepackage[utf8]{inputenc}
\usepackage[croatian]{babel}
\usepackage{listings}
\usepackage{textpos}
\usepackage{algorithm}
\usepackage{algorithmic}
\usepackage{amsthm}
\usepackage{multirow}

\floatname{algorithm}{Algoritam}

%----------------------------------------------------------------------------------------
%	TITLE PAGE
%----------------------------------------------------------------------------------------

\title[]{Određivanje LCP polja korištenjem modificiranog algoritma SA-IS} % The short title appears at the bottom of every slide, the full title is only on the title page

\author[]{Juraj Juričić\\Leon Luttenberger\\Luka Suman} % Your name
\institute[] % Your institution as it will appear on the bottom of every slide, may be shorthand to save space
{
\includegraphics[width=1.7cm]{res/unizg.pdf}\\
Sveučilište u Zagrebu \\
Fakultet elektrotehnike i računarstva \\ 
\medskip
\textit{Projekt iz Bioinformatike}
}
\date{25. siječnja 2019.} % Date, can be changed to a custom date

\begin{document}

\begin{frame}
\titlepage % Print the title page as the first slide
\end{frame}

\begin{frame}
\frametitle{Sadržaj} % Table of contents slide, comment this block out to remove it
\tableofcontents % Throughout your presentation, if you choose to use \section{} and \subsection{} commands, these will automatically be printed on this slide as an overview of your presentation
\end{frame}

%----------------------------------------------------------------------------------------
%	PRESENTATION SLIDES
%----------------------------------------------------------------------------------------

%------------------------------------------------
\section{Uvod} 
%------------------------------------------------

\begin{frame}
\frametitle{Uvod}

\begin{itemize}
	\item Zadatak: izračunati sufiksno polje $SA$ i $LCP$ polje za zadani tekst $T$
	\item Sufiks $S_{i}$ je podniz u $T$ od pozicije $i$ do kraja teksta
	\item $SA$ sadrži leksikografski sortirane indekse svih sufiksa u $T$ 
	\item LCP (\textit{longest common prefixes}) sadrži broj zajedničkih znakova na početku sufiksa čiji su indeksi susjedni u $SA$
\end{itemize}

\begin{center}
	\begin{tabular}{r | c c c c c c c c c c |}
		$i$ & 0 & \textbf{1} & 2 & 3 & \textbf{4} & 5 & 6 & 7 & 8 & 9 \\ \hline
		$T[i]$ & b & a & a & b & a & a & b & a & c & \$ \\ \hline
		$SA[i]$ & 9 & \textbf{1} & \textbf{4} & 2 & 5 & 7 & 0 & 3 & 6 & 8 \\ \hline
		$LCP[i]$ & 0 & 0 & \textbf{4} & 1 & 3 & 1 & 0 & 5 & 2 & 0 \\ \hline
	\end{tabular}
\end{center}

\begin{itemize}
	\item $SA$ i $LCP$ su važni prilikom indeksiranja teksta
\end{itemize}

\end{frame}

%------------------------------------------------

\begin{frame}
\frametitle{Algoritam SA-IS}

\begin{itemize}
	\item Algoritam SA-IS (\textit{SA induced sorting}) klasificira sufikse u $L$ i $S$ tipove
	\item $S_{i}$ je $S$ tipa ako je leksikografski manji od prethodnog sufiksa $S_{i-1}$
	\item koristi se sortiranje po prvim znakovima sufiksa (\textit{bucket sorting})
	\item zatim se inducira (poredak od $S_{i}$ uvjetuje poredak od $S_{i-1}$)
	\item cijeli algoritam se rekurzivno poziva
	\item linearna vremenska i prostorna složenost
\end{itemize}

\end{frame}

%------------------------------------------------

\begin{frame}
\frametitle{Modifikacija za izračun LCP polja}

\begin{itemize}
	\item SA-IS se modificira kako bi se istodobno računalo i $LCP$ polje
	\item brže od linearnih algoritama koji iz gotovog $SA$ računaju $LCP$
	\item ideja je da se $LCP$ vrijednost dvaju sufiksa može zaključiti (inducirati) iz $LCP$ vrijednosti sufiksa koji su uzrokovali njihovo sortiranje
\end{itemize}

\end{frame}

%------------------------------------------------
\section{Mjerenja i rezultati} 
%------------------------------------------------

\begin{frame}
\frametitle{Mjerenja i rezultati}


\begin{table}[]
\resizebox{\textwidth}{!}{
\begin{tabular}{|l|l|l|l|l|l|l|}
\hline
\thead{Ulazni skup}             & \thead{Veličina \\ datoteke} & \thead{Naša \\ impl.} & \thead{Orig. \\ impl.} & \thead{Novija \\ impl.} & \thead{Razlika} &                               \\ \hline
\multirow{2}{*}{E. coli} & \multirow{2}{*}{4.797.285}
   & 1,16  & 0,60  & 0,61  & 92\%  & \thead{Vrijeme (s)} \\ \cline{3-7} 
 & & 85,77 & 43,18 & 43,18 & 98\%  & \thead{Memorija (MB)} \\ \hline
\multirow{2}{*}{A. hydrogenalis} & \multirow{2}{*}{2.482.366}
   & 0.38  & 0,24  & 0,23  & 63\%  & \thead{Vrijeme (s)} \\ \cline{3-7} 
 & & 43,45 & 22,35 & 22,35 & 94\%  & \thead{Memorija (MB)} \\ \hline
\multirow{2}{*}{N. crassa} & \multirow{2}{*}{575.384}
   & 0,073 & 0,057 & 0,056 & 28\%  & \thead{Vrijeme (s)} \\ \cline{3-7} 
 & & 10,28 & 5,18  & 5,18  & 94\%  & \thead{Memorija (MB)} \\ \hline
\multirow{2}{*}{Synthentic A-E} & \multirow{2}{*}{1.000.000}
   & 0,20  & 0,14  & 0,13  & 43\%  & \thead{Vrijeme (s)} \\ \cline{3-7} 
 & & 18,97 & 9,00  & 9,00  & 110\% & \thead{Memorija (MB)} \\ \hline
\multirow{2}{*}{Synthentic A-Z} & \multirow{2}{*}{500.000}
   & 1,51  & 0,63  & 0,63  & 139\%  & \thead{Vrijeme (s)} \\ \cline{3-7} 
 & & 105,2 & 45,00 & 45,00 & 134\%  & \thead{Memorija (MB)} \\ \hline
\multirow{2}{*}{Shakespeare} & \multirow{2}{*}{5.458.199}
   & 1,29  & 0,55  & 0,55  & 134\%  & \thead{Vrijeme (s)} \\ \cline{3-7} 
 & & 100,5 & 49,13 & 49,13 & 105\%  & \thead{Memorija (MB)} \\ \hline
\end{tabular}
}
\caption{Rezultati mjerenja}
\label{mjerenja}
\end{table}

\end{frame}

%------------------------------------------------

\begin{frame}
\frametitle{Literatura}
\footnotesize{
\begin{thebibliography}{99}% Beamer does not support BibTeX so references must be inserted manually as below
\bibitem[new()]{newimpl}
Induced suffix array and lcp construction based on the sais algorithm: new
  source code.
\newblock \url{https://github.com/kurpicz/sais-lite-lcp}.
\newblock Accessed: January 12, 2019.

\bibitem[ori()]{origimpl}
Inducing the {LCP}-array: original source code.
\newblock \url{http://algo2.iti.kit.edu/english/1828.php}.
\newblock Accessed: January 12, 2019.

\bibitem[sha()]{shakespeare}
Shakespeare's complete works txt.
\newblock
  \url{https://ocw.mit.edu/ans7870/6/6.006/s08/lecturenotes/files/t8.shakespeare.txt}.
\newblock Accessed: January 12, 2019.

\bibitem[Fischer(2011)]{fischer2011inducing}
Johannes Fischer.
\newblock Inducing the lcp-array.
\newblock U \emph{Workshop on Algorithms and Data Structures}, stranice
  374--385. Springer, 2011.

\bibitem[Kasai et~al.(2001)Kasai, Lee, Arimura, Arikawa, i
  Park]{kasai2001linear}
Toru Kasai, Gunho Lee, Hiroki Arimura, Setsuo Arikawa, i Kunsoo Park.
\newblock Linear-time longest-common-prefix computation in suffix arrays and
  its applications.
\newblock U \emph{Annual Symposium on Combinatorial Pattern Matching}, stranice
  181--192. Springer, 2001.
\end{thebibliography}
}
\end{frame}

\end{document} 